\def\allfiles{}
\documentclass{package/fancy-book}
\setlength{\parindent}{2em}
\usepackage[UTF8]{ctex}
%%%%%%%%%% Default Package %%%%%%%%%%%%%
\usepackage{package/color-env}
\usepackage{package/quiver}
\usepackage{background}
\usepackage[object=vectorian]{pgfornament} %% used in title.tex
\usepackage{calligra} %%% (optional) to make the Title text beautiful 
\usepackage{lipsum}  %% for dummy text 
\usepackage{amssymb,amsmath,amsfonts}  %%% for maths
\usepackage{datetime}

%%%%%% Optional Packages %%%%%%%
\usepackage{lettrine} %% for nice looking 
\usepackage{GoudyIn} %% first Letter of the paragraph
\renewcommand{\LettrineFontHook}{\color{black}\GoudyInfamily{}}
\LettrineTextFont{\itshape}
\setcounter{DefaultLines}{3}%
%%%%%%%%%%%%%%%%%%%%%%%%%%%%%%%%%%%%%
\usepackage{datetime2}
\usepackage{fourier-orns}
\newcommand{\ornamento}{\vspace{2em}\noindent \textcolor{darkgray}{\hrulefill~ \raisebox{-2.5pt}[10pt][10pt]{\leafright \decofourleft 
\decothreeleft  \aldineright \decotwo \floweroneleft \decoone   \floweroneright 
\decotwo \aldineleft\decothreeright \decofourright \leafleft} ~  \hrulefill \\ \vspace{2em}}}



%%%% Bibliography %%%%%%%%%
% Required packages are included in notes class
% Can be tweaked in the notes.cls file itself
\addbibresource{resource/references.bib}
\includeonly{title,1,2,3}
\begin{document}
\include{title.tex}
\backgroundsetup{contents={}} %% to remove background and watermark from other pages
\tableofcontents

\quad
\newpage
这是笔者于2023年本科四年级下学期学习Morse理论的学习笔记。

我们假定大家拥有基础的微分几何知识和黎曼几何知识。

设$f$是流形$M$上的光滑函数。我们定义:称一个点$p\in M$是$f$的临界点(critical point),若诱导映射$f_*:T_pM \to T_{f(p)}R$是$0$映射。

在流形上我们最好用各种各样的局部坐标讨论。设$(U;x_i,1\leq i \leq n)$是$p$附近的一个局部坐标系,则临界点的定义可以写为:
\begin{align}
	\pa{f}{x^1}=\pa{f}{x^2}=\dots=\pa{f}{x^n}=0
\end{align}

此时$f(p)$称为$f$的临界值。

在临界点处$f$的性质有着与非临界值完全不同的性质。Morse理论则是研究临界点处,$M$本身拓扑性质的改变的理论。
\ifx\allfiles\undefined

	% 如果有这一部分另外的package,在这里加上
	% 没有的话不需要
	
	\begin{document}
\else
\fi
\chapter{流形上的非退化光滑函数}
\section{定义和引理}
我们先用一个引理说明在非临界点$M$的平凡性质。
\begin{lemma}[非临界点]
	设$M^a=\{p\in M|f(p)\leq a\}$。若$a$不是临界值,则$M^a$是带边的光滑流形。
\end{lemma}

引理的证明留作练习。主要使用到隐函数定理以及带边流形的定义。

\begin{definition}
	1
\end{definition}
\ifx\allfiles\undefined
	
	% 如果有这一部分的参考文献的话,在这里加上
	% 没有的话不需要
	% 因此各个部分的参考文献可以分开放置
	% 也可以统一放在主文件末尾。
	
	%  bibfile.bib是放置参考文献的文件,可以用zotero导出。
	% \bibliography{bibfile}
	
	\end{document}
	\else
	\fi
\include{2}
\ifx\allfiles\undefined

	% 如果有这一部分另外的package,在这里加上
	% 没有的话不需要
	
	\begin{document}
\else
\fi
\chapter{Morse不等式的解析证明}
本节我们讨论Witten在上个世纪80年代关于Morse不等式给出的解析证明。%我们讲通过研究流形上的上同调理论以给出Morse不等式,同时讨论这个证明在几何与拓扑两个领域上的重大影响。(论文里面写)
\section{Witten形变与Hodge定理}
设$M$是紧流形.对于$0\leq  i\leq n$的任意整数$i$,令$\beta_i$表示$M$的第$i$个Betti数$\dim (\HdR{1}(M;\R))$。下面的De rham定理说明我们可以用$\beta_i$表示Morse不等式。
\begin{theorem}[De rham定理]
	光滑流形$M$的Derham上同调和奇异上同调存在自然的同构。
\end{theorem}
\section{算子在临界点的分析}
\section{Morse不等式的证明}


\ifx\allfiles\undefined
	
	% 如果有这一部分的参考文献的话,在这里加上
	% 没有的话不需要
	% 因此各个部分的参考文献可以分开放置
	% 也可以统一放在主文件末尾。
	
	%  bibfile.bib是放置参考文献的文件,可以用zotero导出。
	% \bibliography{bibfile}
	
	\end{document}
	\else
	\fi
\end{document}