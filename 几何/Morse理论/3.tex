\ifx\allfiles\undefined

	% 如果有这一部分另外的package,在这里加上
	% 没有的话不需要
	
	\begin{document}
\else
\fi
\chapter{Morse不等式的解析证明}
本节我们讨论Witten在上个世纪80年代关于Morse不等式给出的解析证明。%我们讲通过研究流形上的上同调理论以给出Morse不等式,同时讨论这个证明在几何与拓扑两个领域上的重大影响。(论文里面写)
\section{Witten形变与Hodge定理}
设$M$是紧流形.对于$0\leq  i\leq n$的任意整数$i$,令$\beta_i$表示$M$的第$i$个Betti数$\dim (\HdR{1}(M;\R))$。下面的De rham定理说明我们可以用$\beta_i$表示Morse不等式。
\begin{theorem}[De rham定理]
	光滑流形$M$的Derham上同调和奇异上同调存在自然的同构。
\end{theorem}
\section{算子在临界点的分析}
\section{Morse不等式的证明}


\ifx\allfiles\undefined
	
	% 如果有这一部分的参考文献的话,在这里加上
	% 没有的话不需要
	% 因此各个部分的参考文献可以分开放置
	% 也可以统一放在主文件末尾。
	
	%  bibfile.bib是放置参考文献的文件,可以用zotero导出。
	% \bibliography{bibfile}
	
	\end{document}
	\else
	\fi