\documentclass[UTF8]{ctexart}[a4paper,10pt]
\usepackage[thmmarks]{ntheorem}
\usepackage{amsmath}
\usepackage{amsfonts,amssymb} 
\usepackage{thmtools}
\usepackage[hmargin=2.5cm,vmargin=2.5cm]{geometry}
\usepackage{tikz-cd,tikz}
\usepackage{graphicx,float}
\usepackage{fancyhdr}
\usepackage{fourier-orns}
\usepackage{quiver}

%声明环境
\theorembodyfont{\rmfamily}
\newtheorem{example}{例}[section]              
\newtheorem{algorithm}{算法}[subsection]
\newtheorem{theorem}{定理}[section]            
\newtheorem{definition}{定义}[section]
\newtheorem{axiom}{公理}[section]
\newtheorem{property}{性质}[section]
\newtheorem{proposition}{命题}[section]
\newtheorem{lemma}[theorem]{引理}
\newtheorem{corollary}[theorem]{推论}
{
    \theoremheaderfont{\sffamily}
    \newtheorem*{remark}{注解} 
}
\newtheorem{condition}{条件}
\newtheorem{conclusion}{结论}[section]
\newtheorem{assumption}{假设}
{
\theoremstyle{nonumberplain}
\theoremheaderfont{\bfseries}
\theorembodyfont{\normalfont}
\theoremsymbol{\mbox{$\Box$}}
\newtheorem{proof}{证明}
}
%定义命令
\def\N{\mathbb{N}}
\def\Z{\mathbb{Z}}
\def\Q{\mathbb{Q}}
\def\R{\mathbb{R}}
\def\C{\mathbb{C}}
\def\S{\mathbb{S}}
\def\D{\mathbb{D}}
\def\H{\mathbb{H}}
%外测度
\newcommand{\pa}[3][]{\dfrac{\partial^{#1} #2}{\partial #3^{#1}}}
\newcommand{\II}{\mathrm{II}}
%页眉设计
\renewcommand 
\headrule{
\hrulefill
\raisebox{-2.1pt}
{\quad{\FourierOrns M T S N}\quad}
\hrulefill}
\pagestyle{fancy}

%超链接红色
\usepackage[colorlinks,linkcolor=red]{hyperref}

\usepackage{enumerate}


\title{Note For Complex Geometry}
\author{整理者:Tsechi/Tseyu}
\begin{document}
\maketitle
\tableofcontents
\section{复几何初探}
本章的主要来源是陈先生在教材《微分几何》中复流形一章的内容。我们将给出一些基础的概念。
\subsection{复流形的定义}
\begin{definition}
    设$M$是具有可数基的Hausdorff空间。若在$M$上给定了坐标卡$\{(U_\alpha,\phi_\alpha)\}$使得$\{U_\alpha\}$是开覆盖,且$\varphi_\alpha$是从$U_\alpha$到$\C^m$的同胚,并且满足转移映射是$\C_m$之间的全纯映射,则称$M$是$m$维复流形。
\end{definition}
多元复函数全纯意指:
\begin{proposition}
    下面三个条件是彼此等价的。

    1.$f$是多元全纯函数。(对每个分量都满足柯西黎曼方程)。

    2.对于任意$a \in U$,存在邻域$V \subset U$使得$f$在$V$可以表示为收敛的幂级数:
    \begin{align*}
        f(z)=\sum_{k_1,\dots,k_m=0}^\infty c_{k_1,\dots,k_m}(z^1-a^1)^{k_1}\dots (z^m-a^m)^{k_m}
    \end{align*}

    3.复导数$\pa{f}{z}$在$U$中处处存在。(对于任何分量$z^k$)

\end{proposition}

考虑转移函数$\phi_\beta \circ \phi_\alpha^{-1}$。其是同胚,从而Jacobi矩阵:
\begin{align}
    \frac{\partial(w^1,\dots,w^m)}{\partial(z^1,\dots,z^m)} \neq 0
\end{align}
实际上,我们可以计算$m=2$的情况,以思考该矩阵的行列式和4维实矩阵行列式的关系。该矩阵的行列式的模长实际上是对应的是实$2m$维的Jacobi行列式。

设$f:M \to \C$是复流形上的全纯函数(限制在每个坐标邻域上都是全纯的)。根据极大模原理,若$p_0 \in M$的一个邻域$U$内$f$在$p_0$的模取得最大值,即$|f(p)| \leq |f(p_0)|$,则在$U$内有:
\begin{align*}
    f(p)=f(p_0)
\end{align*}
如果$M$是紧致的连通复流形,$|f(p)|(p \in M)$是$M$上的连续函数。他必然在$M$上取得最大值,于是全纯函数$f$必然取常数。

\begin{example}
    $\C_m$是$m$维复流形,$\C_1$是Gauss复平面。
\end{example}
\begin{example}
    复$m$维射影空间$\C P^m$。

    构造这个空间的复结构的方法和$\R P^m$的方法一模一样。在变换中,由于转换映射是分式,所以全纯函数。(分母不等于$0$)

    考虑一些$m$较小的情况。如$m=1$,此时$\C P^1$被两个坐标邻域$U_0,U_1$覆盖。因此$\C P^1$是Gauss复平面的一点紧化:$S^2$。

    考虑自然的投影$\pi:\C_{m+1}-\{0\} \to \C P^m$。则对于$p \in \C P^m$有$\pi^{-1}(p)$与$C^*=\C-\{0\}$等同。显然根据局部坐标,有:
    \begin{align*}
        \pi^{-1}(U_j)=U_j \times C^*
    \end{align*}
    可以验证转移函数都是全纯的。因此$\C^m$是$\C P^m$上全纯的纤维丛。

    我们可以把上述纤维做一些变化。考虑$S^{2m+1}$到$\C P^m$的投影。则纤维转化为$S^1$。这称为Hopf纤维化。

    当$m=1$,$\pi: S^3 \to S^2$是一个重要的映射。这是一个非零伦的映射。因为根据同伦正合,$\pi_n(S^3)$和$\pi_n(S^2)$是由$\pi$给出的同构。
\end{example}
\begin{example}
    由一组齐次多项式$P(z^1,\dots,z^m)=0$给出的轨迹在$\C P^m$中被称为代数流形。周炜良的一个定理说,隐蔽在$\C P^m$的每一个紧致子流形必是一个代数流形。
\end{example}
\begin{example}[复环面]
    将$\C^m$看作$\R^{2m}$所得到的实流形具有复结构,称为$m$维复环面。

    复结构给出了复环面更多的性质。如$m=1$时,复环面到自身的全纯映射是保角的。

    若复环面可以嵌入到复射影空间作为非奇异子流形,即对于充分大的$N$,存在非退化的全纯映射:
    \begin{align*}
        f:\C_m/L \to \C P^N
    \end{align*}
    则称该复流形为Abel流形。
\end{example}
\begin{example}[Hopf流形]
    考虑$\C_m-\{0\} \to \C_m -\{0\}$,将$(z^1,\dots,z^m)$映射为$2(z^1,\dots,z^m)$。商空间$\C^m-\{0\}$商去这个映射生成的等价关系得到的流形是$m$维复流形。其与$S^{2m-1}\times S^1$同胚。

    Hopf流形是最简单的非代数流形。
\end{example}
\begin{example}[黎曼曲面]
    设$M$是二维定向曲面,有度量:
    \begin{align*}
        ds^2=(\omega_1)^2+(\omega_2)^2
    \end{align*}
    假定$ds^2$是解析的,则:
    \begin{align*}
        ds^2=(\omega_1+i\omega_2)(\omega_1-i \omega_2)
    \end{align*}
    可以对$\omega_1+i\omega_2$积分,得到:
    \begin{align*}
        dz=\lambda(\omega_1+i\omega_2)
    \end{align*}
    于是$ds^2=\dfrac{1}{|\lambda|^2}dz d\bar{z}=\dfrac{1}{|\lambda|^2}(dx^2+dy^2)$

    二维定向曲面必定有复流形构造,使其成为一维复流形。这称之为黎曼曲面。
\end{example}
\subsection{复结构与近复结构}
这一节我们简要说明,当给出实线性空间$V$的时候,如何定义上面的复结构。从而如果实流形的切空间总能有复结构的时候,称之为其上的近复结构。
\subsubsection{线性空间的复结构}
\begin{definition}
    设$V$是$m$维实线性空间。所谓$V$上的复结构是指线性变换$J:V \to V$,使得$J^2=-\mathrm{id}$。如果$V$上有复结构$J$,可以自然诱导$V^*$上的复结构:
    \begin{align*}
        \langle x, J\alpha\rangle:=\langle Jx,\alpha\rangle
    \end{align*}

    容易验证$J^2 \alpha=-\alpha$。
\end{definition}
 
 选取$V$的一组基底$e_r$,设$J$对应的矩阵$A=(a_i^j)$。于是$A^2=-I$。于是矩阵有最小多项式$x^2+1=0$。因此$A$的特征值只可能为$i,-i$,且必须成对出现。因此$m=2n$是偶数。

 根据线性代数知识,若$e^{*r}$是对偶的$V^*$基底,则可以得到:$J(e^{*r})^{t}={}^t A (e^{*r})^{t}$.所以$V^*$上的$J$矩阵拥有与$A$相同的特征值。

 现在考虑$V^*$的复化:$V^*\otimes_{\R} \C$。显然复化后,其中的元素可以写为:$\lambda=\alpha+i \beta$.于是$V^*$的基底自然成为复线性空间$V^*\otimes_{\R} \C$的基底。$V^*$的复结构也可以延拓到$V^*\otimes_{\R} \C$上.

 现在复化空间上必然存在$i$和$-i$的特征向量。$i$的特征向量称为$(1,0)$元素,而$-i$的特征向量称为$(0,1)$元素。显然全体$(1,0)$元素组成了子复空间,记作$V_c$.另一个则记为$\overline{V_{C}}$。我们断言,把$(1,0)$型向量取共轭,会得到$(0,1)$向量。这里省略断言的证明。

 另一方面,任何一个复向量都可以写为两类向量的和。因此这是一个直和。
 \begin{align*}
    f=f_1+f_2, f_1=\frac{1}{2}(f-iJf),f_2=\frac{1}{2}(f+iJf)
 \end{align*}
 所以两个空间都是$m/2$维复向量空间。

 现在考虑一些新的东西。在子空间$V_C$上选取一组基底,以及与之对应的共轭基底。在这组基下,$J$的矩阵为$\mathrm{diag}(i,i,\dots,i,-i,-i,\dots,-i)$。

 现在考虑$V$上的复值线性函数。设$\lambda^i$是$V_c$的基,则设$\lambda^i=e^{*i}+ie^{*n+i}$.根据$\lambda^i$对应的特征值是$i$,我们有:
 \begin{align*}
    Je^{*j}=-e^{*n+j}, Je^{*n+j}=e^{*j}
 \end{align*}
 显然$e^{*j}$和$e^{*n+j}$可以生成$\lambda^j$和$\lambda^j$的共轭,因此他们也生成了整个$V^*$。换句话说,是$V^*$的一组基。

 设$e_j,e_{n+j}$是$V$中对偶上述基的基。则:
 \begin{align*}
    Je_j=e_{n+j}, Je_{n+j}=-e_j
 \end{align*}
 \begin{theorem}
    拥有复结构的实向量空间一定是偶数维的。并且这样的空间可以赋予基$\{e_i,Je_j\}$.此外,任何两个这样的基底赋予$V$相同的定向。
 \end{theorem}
 \begin{proof}
    我们只需要证明定向。考虑$V^*$的基底$e^{*j},-Je^{*j}$.则有:
    \begin{align*}
        -e^{*j}\wedge e^{*j+n}=-\frac{i}{2}\lambda^j\wedge \overline{\lambda^j}
    \end{align*}
    从而把所有基楔积,得到$(\dfrac{-i}{2})^n \bigwedge_{i=1}^n (\lambda^i\wedge \overline{\lambda^i})$.

    现在考虑$\mu^j,\overline{\mu^j}$生成了另外一组基。则存在$n \times n$非退化矩阵$G$:
    \begin{align*}
        (\mu^i)=(\lambda^i)G
    \end{align*}
    因此
    \begin{align*}
        \bigwedge_{i=1}^n (\lambda^i\wedge \overline{\lambda^i})=|\mathrm{det}G|^2\bigwedge_{i=1}^n (\mu^i\wedge \overline{\mu^i})
    \end{align*}
    这说明给出的定向是一致的。
 \end{proof}
 \begin{proposition}
    设$V$是实向量空间。若$V^* \otimes \C$有任意直和分解,使得在复共轭关系下有一一对应,则存在唯一的复结构$J$使得以$V_C$为$(1,0)$型,另外一个为$(0,1)$型。
 \end{proposition}
 \begin{proof}
    对于$f \in V^* \otimes \C$,我们定义$J$为$Jf=if$,若$f \in V_C$。若$f \in \overline{V_c}$,则定义为$Jf=-if$。
 \end{proof}
\end{document}