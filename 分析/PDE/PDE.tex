\documentclass[UTF8]{ctexart}[a4paper,12pt]
\usepackage[thmmarks]{ntheorem}
\usepackage{amsmath}
\usepackage{amsfonts,amssymb} 
\usepackage{thmtools}
\usepackage[hmargin=2.5cm,vmargin=2.5cm]{geometry}
\usepackage{tikz-cd,tikz}
\usepackage{graphicx,float}
\usepackage{fancyhdr}
\usepackage{fourier-orns}
\usepackage{quiver}

%声明环境
\theorembodyfont{\rmfamily}
\newtheorem{example}{例}[section]              
\newtheorem{algorithm}{算法}[subsection]
\newtheorem{theorem}{定理}[section]            
\newtheorem{definition}{定义}[section]
\newtheorem{axiom}{公理}[section]
\newtheorem{property}{性质}[section]
\newtheorem{proposition}{命题}[section]
\newtheorem{lemma}[theorem]{引理}
\newtheorem{question}{问题}[section]
\newtheorem{corollary}[theorem]{推论}
{
    \theoremheaderfont{\sffamily}
    \newtheorem*{remark}{注解} 
}
\newtheorem{condition}{条件}
\newtheorem{conclusion}{结论}[section]
\newtheorem{assumption}{假设}
{
\theoremstyle{nonumberplain}
\theoremheaderfont{\bfseries}
\theorembodyfont{\normalfont}
\theoremsymbol{\mbox{$\Box$}}
\newtheorem{proof}{证明}
}
%定义命令
\def\N{\mathbb{N}}
\def\Z{\mathbb{Z}}
\def\Q{\mathbb{Q}}
\def\R{\mathbb{R}}
\def\C{\mathbb{C}}
\def\S{\mathbb{S}}
\def\D{\mathbb{D}}
\def\H{\mathbb{H}}
\def\F{\mathbb{F}}


%页眉设计
\renewcommand 
\headrule{
\hrulefill
\raisebox{-2.1pt}
{\quad{\FourierOrns M T S N}\quad}
\hrulefill}
\pagestyle{fancy}

%超链接红色
\usepackage[colorlinks,linkcolor=red]{hyperref}

\usepackage{enumerate}


\title{PDE知识整理}
\author{整理者:颜成子游/南郭子綦}
\begin{document}
\maketitle
\tableofcontents
1.齐次化原理。

设$L$是$t,x$的线性微分算子。且关于$t$的最高阶求导次数小于$m$,则定解问题:
\begin{equation*}
    \left \{\begin{aligned}
        &\frac{\partial^m \omega}{\partial t^m}=L\omega+f(x,t),t >0\\
        &\omega|_{t=0}=\dots=\frac{\partial^{m-1} \omega}{\partial t^{m-1}}|_{t=0}=0
    \end{aligned}\right.
\end{equation*}
的解表示为:
\begin{align*}
    \omega=\int_0^t z(x,t;\tau)d\tau
\end{align*}
其中$z$是定解问题:
\begin{equation*}
    \left \{\begin{aligned}
        &\frac{\partial^m z}{\partial t^m}=Lz,t >\tau>0\\
        &z|_{t=\tau}=\dots=\frac{\partial^{m-2} z}{\partial t^{m-2}}|_{t=\tau}=0,\frac{\partial^{m-1} z}{\partial t^{m-1}}|_{t=\tau}=f(x,\tau)
    \end{aligned}\right.
\end{equation*}
的解。

在实际应用中,如果有第一个方程待解,可以考虑下面的方程。把$\tau$看作参数般解出$z$后,对$\tau$做积分得到第一个方程的解。

\begin{example}[Duhamel原理:对于波动方程]
    设柯西问题为:
    \begin{equation*}
        \left\{\begin{aligned}
            &\square_4 \omega=0, t>\tau\\
            &\omega(x,\tau,\tau)=0\\
            &\omega_t(x,\tau,\tau)=f(x,\tau)
        \end{aligned} \right.
    \end{equation*}

    则$u(x,t)=\int_0^t \omega(x,t,\tau)d\tau$是柯西问题:
    \begin{equation*}
        \left\{\begin{aligned}
            &\square_4 u=f(x,t),t>0\\
            &u(\dot{x},0)=0\\
            &u_t(x,0)=0
        \end{aligned}\right.
    \end{equation*}
    的解。只需要把$L$代为$\Delta$算子即可。
\end{example}
\end{document}
