\documentclass[UTF8]{ctexart}[a4paper,10pt]
\usepackage[thmmarks]{ntheorem}
\usepackage{amsmath}
\usepackage{amsfonts,amssymb}
\usepackage{thmtools}
\usepackage[hmargin=2.5cm,vmargin=2.5cm]{geometry}
\usepackage{tikz-cd,tikz}
\usepackage{graphicx,float}
\usepackage{fancyhdr}
\usepackage{fourier-orns}

%声明环境
\newtheorem{example}{例}[section]
\newtheorem{algorithm}{算法}[subsection]
\newtheorem{theorem}{定理}[subsection]
\newtheorem{definition}{定义}[section]
\newtheorem{axiom}{公理}[section]
\newtheorem{property}{性质}[section]
\newtheorem{proposition}{命题}[section]
\newtheorem{lemma}[theorem]{引理}
\newtheorem{corollary}[theorem]{推论}
{
    \theoremheaderfont{\sffamily}
    \newtheorem*{remark}{注解}
}
\newtheorem{condition}{条件}
\newtheorem{conclusion}{结论}[section]
\newtheorem{assumption}{假设}
{
\theoremstyle{nonumberplain}
\theoremheaderfont{\bfseries}
\theorembodyfont{\normalfont}
\theoremsymbol{\mbox{$\Box$}}
\newtheorem{proof}{证明}
}
%定义命令
\def\N{\mathbb{N}}
\def\Z{\mathbb{Z}}
\def\Q{\mathbb{Q}}
\def\R{\mathbb{R}}
\def\C{\mathbb{C}}
\def\S{\mathbb{S}}
\def\D{\mathbb{D}}
\def\H{\mathbb{H}}
%外测度
\def\outmQ{m_*(Q)}

%页眉设计
\renewcommand
\headrule{
\hrulefill
\raisebox{-2.1pt}
{\quad{\FourierOrns M T S N}\quad}
\hrulefill}
\pagestyle{fancy}

%超链接红色
\usepackage[colorlinks,linkcolor=red]{hyperref}

\usepackage{enumerate}


\title{Complex Analysis Stduy Notes}
\author{颜成子游}
\begin{document}
\maketitle
\tableofcontents
\section{复分析的基础前置}
这一章主要是列举一些关于复数,复函数的前置知识。以及复平面作为拓扑结构,沿路径的复函数积分等等定义。很简单,只列举基本概念和基础定理,不做证明。
\subsection{复数和复平面}
首先要清楚复数的基本运算:加减乘除。

复数的模长和基础的三角不等式。

实部与虚部。注意实部可以用$z+\overline{z}$的一半来表示。

复数的收敛与实数类似:
$$
\lim_{n\to +\infty}|z_n-\omega|=0
$$
即距离趋近于0.

\begin{theorem}
    复平面是完全的。
\end{theorem}
作为复平面上的柯西列,必然分量也是柯西列。实数是完全的,从而复平面也是完全的。

现在考察复平面的拓扑结构。与$\R^2$的结构是极其类似的。由于点集拓扑学已经详细的研究了$\R^2$的拓扑性质,我们不再赘述。

\subsection{复平面上的函数}
复数可以由有序数对$(x,y)$唯一决定。某种意义上,复函数是一个向量函数:$s=f(x,y),t=g(x,y)$。但并不能用这样的视角来研究复函数。这样会掩盖掉复函数许多丰富的性质。

首先研究连续的复函数。比照实函数的定义,复函数的连续当然是毋庸置疑的。由于模长大于实部和虚部的绝对值,因此复函数连续也意味着实部函数和虚部函数作为二元函数连续。反之亦然。因此连续的复函数的模长函数也连续。

接着研究可导。与实函数的可导不同,实部函数与虚部函数都可微并不能带来复函数可导。我们先给出定义:
\begin{definition}
    对于复函数$f$,和其定义域$\Omega$。$\Omega$是一个开集。称$f$在$z_0$可导,若极限:
    $$
    \lim_{h \to 0}\frac{f(z_0+h)-f(z_0)}{h}
    $$
    存在。极限值是$f$在$z_0$的导数值,记为$f'(z_0)$。
\end{definition}
    必须注意,$h$可以是任何方向的复数。极限存在意味着任何一个方向的$h$趋近0时,上述式的极限相同。

\begin{definition}
    若$f$在整个$\Omega$都可导,称其是一个$\Omega$上的全纯函数。若$\Omega$是整个复平面,称其为整函数。
\end{definition} 
\begin{example}
        $f(z)=\frac{1}{z}$在除了$z=0$外点都可导。$g(z)=\overline{z}$不可导。因为纯虚数和实数的$h$极限不同。
\end{example}
由于极限形式完全相同,从而复函数的求导四则运算与实函数一模一样。
\end{document}